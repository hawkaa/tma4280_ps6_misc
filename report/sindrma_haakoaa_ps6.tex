\documentclass{article}

\usepackage{fancyhdr} % Required for custom headers
\usepackage[utf8]{inputenc}
\usepackage{lastpage} % Required to determine the last page for the footer
\usepackage{extramarks} % Required for headers and footers
\usepackage[usenames,dvipsnames]{color} % Required for custom colors
\usepackage{graphicx} % Required to insert images
\usepackage{listings} % Required for insertion of code
\usepackage{color}
\usepackage{courier} % Required for the courier font
\usepackage{lipsum} % Used for inserting dummy 'Lorem ipsum' text into the template
\usepackage[norsk]{babel}

% Margins
\topmargin=-0.45in
\evensidemargin=0in
\oddsidemargin=0in
\textwidth=6.5in
\textheight=9.0in
\headsep=0.25in

\linespread{1.1} % Line spacing

% Set up the header and footer
\pagestyle{fancy}

\lhead{\exerciseGroup} % Top left header
\chead{\exerciseClass: \exerciseTitle} % Top center head
\rhead{\firstxmark} % Top right header
\lfoot{\lastxmark} % Bottom left footer
\cfoot{} % Bottom center footer
\rfoot{Page\ \thepage\ of\ \protect\pageref{LastPage}} % Bottom right footer
\renewcommand\headrulewidth{0.4pt} % Size of the header rule
\renewcommand\footrulewidth{0.4pt} % Size of the footer rule

\setlength\parindent{0pt} % Removes all indentation from paragraphs


% Document data

\newcommand{\exerciseTitle}{Øving 6} % Assignment title
\newcommand{\exerciseClass}{TMA4280} % Course/class
\newcommand{\exerciseGroupMembers}{Sindre Magnussen og Håkon Åmdal}
%----------------------------------------------------------------------------------------
%	TITLE PAGE
%----------------------------------------------------------------------------------------
\newcommand{\HRule}{\rule{\linewidth}{0.5mm}}
\title{
\vspace*{\stretch{1}}
\noindent\HRule
\begin{center}
 \Huge
 \noindent	\exerciseClass \\
 \noindent \exerciseTitle \\ [4mm]
 \large
 \noindent\emph{\exerciseGroupMembers}
\noindent\HRule \newline
\end{center}
\vspace{0cm}
\begin{center}
	\includegraphics[width=10cm]{img/kongull.jpg}
\end{center}
\vspace*{\stretch{3}}
\begin{center}
\end{center}
}
% Insert date here if you want it to appear below your name

% C code

\begin{document}
\pagestyle{empty}
\maketitle

\thispagestyle{empty}

\newpage \tableofcontents


\newpage

\pagenumbering{arabic}

\section{Introduksjon - Poisson-problemet}
Litt om problemet

\section{Mulige løsningsstragegier}
Beskrivelse av poisson-problemet generelt.
\subsection{Iterative}
\subsection{Absolutte}
Sammenlign med vanlig LU-faktorisering. Ikke så veldig mye , men nok til å sette det i perspektiv.

\section{Beskrivelse av løsning}
Vi har delt det opp, og har et felles bibilotek.

\subsection{Ulike programkomponenter}
\subsubsection{poisson}
\subsubsection{convergence}

\subsection{Testing}

\section{Tester og resultater}
Med unntak av konvergenstestene, så ble testene gjennomført ved å kjøre programmet ti ganger. Den største og minste kjøretiden ble fjernet, og gjennomsnittet ble regnet ut. Koden for testene under ble kompilert med de samme instillingene, det vil si at både MPI og OpenMP var aktivert. 
Det vil si at vi kjørte programmene med \emph{mpirun} og heller varierte parameterene vi sendte med til kjøringen. Vi ville heller kjøre testene på en ekvivalent måte, enn å ha forsjellige måter å kjøre programmet på gitt forskjellige parametre. 

\subsection{Konvergens med ulik p}

\subsection{Kjøretid med forskellige p*t = 36}
Denne testen ble gjennomført for noen kombinasjoner som gir \emph{p} * \emph{t} = 36. Resultatet er vist i tabell \ref{p/t-table}.

\begin{table}
\begin{center}
	
	\begin{tabular}{c | c | c}
	\hline \hline 
	Antall prosesser      &    Antall tråder     &    Kjøretid (i sekunder) 	    \\ \hline	
	12		      &		1	     &	  32.690540       		    \\ \hline
	6		      &         2	     &    32.975599       		    \\ \hline
	4		      &         3	     &    33.223129	    		    \\ \hline
	3   		      &		2	     &    33.633001	    		    \\ \hline
	2		      &         6	     &    34.279941	    		    \\ \hline
	1		      &		12	     &    35.741616	    		    \\ \hline
	
	\end{tabular}
\end{center}
\caption{Kjøretider gitt forskjellige kompinasjoner av \emph{p} og \emph{t}.}
\label{p/t-table}
\end{table}

Som tabell \ref{p/t-table} viser, er det faktisk negativt å bruke en hybrid-løsning. Vi ser at kjøretiden blir litt større jo flere tråder vi bruker. Grunnen er at det koster å opprette og samle tråder. Så i dette tilfellet lønner det seg dermed å kjøre programmene med kun en tråd. 
Testene som følger denne vil dermed kun bruke en tråd og heller variere antall prosesser. 

\subsubsection{Hvorfor kun MPI er raskest?}

\subsection{Speedup og effektivitet}

\begin{table}
\begin{tabular}{c | c c c c c c c c c}

n/p & 1 & 2 & 4 & 8 & 12 & 24 & 36 & 48 & 96 \\
8 &	5.4001808166503906250000000e-05 & 3.6319096883138023092087859e-05 \\
16 &    2.6587645212809246598670287e-04 & 1.4833609263102212638164856e-04 \\
32 &    1.2506643931070964264468115e-03 & 6.5469741821289062500000000e-04 \\
64 &    5.7186683019002275754627540e-03 & 2.9246807098388671875000000e-03 \\
128 &   2.6107430458068847656250000e-02 & 1.3779799143473306713425508e-02 \\
256 &   1.1788566907246907089490406e-01 & 6.1504205067952476271297968e-02 \\
512 &   5.1830295721689856325298251e-01 & 2.6001699765523272889211626e-01 \\
1024 &  2.3079368273417153467619300e+00 & 1.1539564927419025952559650e+00 \\
2048 &  1.0080118298530578613281250e+01 & 5.0396489699681596974301101e+00 \\
4096 &	4.4799674709637962166652869e+01 & 2.2249505241711933223314190e+01 \\
8192 &	1.9275400197505950927734375e+02 & 9.6106613675753280290336988e+01 \\
16384 &	8.2507074578603112513519591e+02 & 4.1267060649394989013671875e+02 \\











\end{tabular}
\end{table}


\subsection{Forslag til forbedringer}


\end{document}
